\documentclass[conference]{IEEEtran}
\IEEEoverridecommandlockouts
% The preceding line is only needed to identify funding in the first footnote. If that is unneeded, please comment it out.
\usepackage{cite}
\usepackage{amsmath,amssymb,amsfonts}
\usepackage{algorithmic}
\usepackage{graphicx}
\usepackage{textcomp}
\usepackage{xcolor}
\def\BibTeX{{\rm B\kern-.05em{\sc i\kern-.025em b}\kern-.08em
    T\kern-.1667em\lower.7ex\hbox{E}\kern-.125emX}}
\begin{document}

\title{
YZV303(2)E/BLG561E  Deep Learning \\ 
Project Proposal\\
Project Title*
}

\author{\IEEEauthorblockN{Student Name Surname}
\IEEEauthorblockA{\textit{dept. name of organization (of Aff.)} \\
\textit{name of organization (of Aff.)}\\
e-Mail Address\\ Student No}
\and
\IEEEauthorblockN{Student Name Surname}
\IEEEauthorblockA{\textit{dept. name of organization (of Aff.)} \\
\textit{name of organization (of Aff.)}\\
e-Mail Address\\ Student No}
}

\maketitle
\vspace{-1cm}
\section{Project Description}
Explain the project briefly.

\section{Problem Definition}
Explain the problem formally.

\section{Dataset}
Explain the specifications of the dataset you plan to create. Also explain the data collection process.

\section{Methodology}
Describe the method you will use briefly. You don't necessarily need to follow this methodology strictly. Just try to understand what kind of models you can use for your problem and explain it.

% You can use 'printbibliography' with a ref.bib file
% \printbibliography


\begin{thebibliography}{00}
\bibitem{b1} G. Eason, B. Noble, and I. N. Sneddon, ``On certain integrals of Lipschitz-Hankel type involving products of Bessel functions,'' Phil. Trans. Roy. Soc. London, vol. A247, pp. 529--551, April 1955.
\bibitem{b2} J. Clerk Maxwell, A Treatise on Electricity and Magnetism, 3rd ed., vol. 2. Oxford: Clarendon, 1892, pp.68--73.
\end{thebibliography}


\end{document}
